\documentclass[12pt, unicode]{beamer}
\usetheme{Warsaw}
\usepackage{luatexja}
\usepackage{color}
\usepackage{listings}
\lstset{
  basicstyle=\ttfamily\bfseries,
  commentstyle=\color{red}\itshape,
  stringstyle=\color{black!30!green},
  showstringspaces=false,
  keywordstyle=\color{blue}\bfseries
}
%color
\definecolor{battleshipgrey}{rgb}{0.52, 0.52, 0.51}

\title{How to make bindings in Rust}
\subtitle{... and its modern toolchain, Drop traits, and Windows suffering.}
\author{Hiroshi Hatake}
\date[2016/04/08]{Technical information sharing seminar}

\begin{document}

\frame{\maketitle}

\begin{frame}{Introduction}
\begin{block}{What is ``Binding''?}
Binding is a library which call or use other language's function and feature.
\end{block}
\end{frame}

\begin{frame}{Introduction}
\begin{block}{Difficutlies of ``Binding''}
Binding is difficult. Because ...
\end{block}
\onslide<1->
\begin{itemize}
\item<2-> Need to call Foreign Function Interface
\item<3-> Absorb language differences (C and other languages)
\item<4-> Interact two different environment
\item<5-> Control mangling rules
\end{itemize}
\end{frame}

\newcommand\Small{\fontsize{9}{9.2}\selectfont}
\begin{frame}[fragile]{How to create bindings in Rust}
\begin{block}{Rust has generating binding tool from C header}
It is \textbf {rust bindgen}\footnote[frame]{https://github.com/Yamakaky/rust-bindgen}.
\end{block}
\onslide<2->
{\large Example:}
\onslide<2->
\begin{lstlisting}[language={bash},basicstyle=\ttfamily\Small]
  bindgen --link lua \
    --builtins /usr/include/lua.h --output lua.rs
\end{lstlisting}
\end{frame}

\begin{frame}[fragile]{Generate binding from C header}
\begin{block}{Actual example}
Then, let's generate binding for Groonga!
\end{block}
\onslide<2->
\begin{lstlisting}[language={bash},basicstyle=\ttfamily\Small]
  bindgen --link groonga \
    --builtins /usr/include/groonga/groonga.h \
    --output groonga.rs
\end{lstlisting}
\end{frame}

\begin{frame}[fragile]{Create binding crate}
\begin{block}{Actual example}
Second, put into src directory.
\end{block}
\begin{lstlisting}[language={bash},basicstyle=\ttfamily\Small]
% tree -L 2 .
.
├── Cargo.toml
├── build.rs
└── src
    ├── groonga.rs # <- e.g.) Put groonga.rs here!
    └── lib.rs
\end{lstlisting}
\end{frame}

%% Define highlighting for Rust.
\lstdefinelanguage{Rust} {
  morecomment = [l]{//},
  morecomment = [l]{///},
  morecomment = [s]{/*}{*/},
  morestring=[b]",
  sensitive = true,
  morekeywords = {extern, return, crate, fn, let, use, mut, impl, for, struct, pub, if}
}
\begin{frame}[fragile]{Create binding crate}
\begin{block}{Actual example}
Declare using groonga module in lib.rs which is the library entry point.
\end{block}
\begin{lstlisting}[language={Rust},basicstyle=\ttfamily\Small]
extern crate libc;

pub mod groonga;
\end{lstlisting}
\end{frame}

\frame{\centering \Large Then, Complie and fix!}

\frame{\centering \Large Complie and fix!!}

\frame{\centering \Large Complie and fix!!!}

\frame{\centering \Large Until errors are dismissed....}

\begin{frame}[fragile]{Create binding crate}
\begin{block}{Actual example}
Confirm to succeed to be built.
\end{block}
\begin{lstlisting}[language={bash},basicstyle=\ttfamily\Small]
% cargo build
    Updating registry `...`
   Compiling pkg-config v0.3.8
   Compiling libc v0.2.16
   Compiling groonga-sys v0.3.0 (file:///...)
\end{lstlisting}
\onslide<2-> {\Large Yay!}
\end{frame}

\begin{frame}{Traits, manage resources, Rustish concept}
\begin{block}{Make more Rustish}
rust-bindgen does \textbf {not} generate Rustish binding.
\newline
It is auto generated and just as a set of function signature declarations.
\end{block}
\end{frame}

\begin{frame}{Traits, manage resources, Rustish concept}
\begin{block}{Make more Rustish}
  How to make more Rustish binding?
  \newline
  \onslide<2->
  You should know about \textbf {Traits}. Especially, \textbf {Drop trait}.
\end{block}
\end{frame}

\begin{frame}[fragile]{Traits, manage resources, Rustish concept}
\begin{block}{Traits}
  Traits tells a funtionality which must be implemented in type.
  \newline
  \onslide<2->
  It sometimes are used in generics bound.
\end{block}
\begin{lstlisting}[language={Rust},basicstyle=\ttfamily\Small]
  fn from_iter<T: Iterator<A>>(iterator: T)
    -> SomeCollection<A>
\end{lstlisting}
\end{frame}

\begin{frame}[fragile]{Traits, manage resources, Rustish concept}
\begin{block}{Drop trait\footnote[frame]{https://doc.rust-lang.org/std/ops/trait.Drop.html}}
  Drop trait's drop method is called when a variable goes out of scope.\footnote[frame]{Perhaps, C++ users noticed by intuition that Drop trait is similar to concept of destructor.}
\end{block}
\onslide<2->
\lstinputlisting[language={Rust},basicstyle=\ttfamily\Small]{mydrop.rs}
\end{frame}

\begin{frame}[fragile]{Traits, manage resources, Rustish concept}
\begin{block}{Actual example}
In Ruroonga, Drop trait is often used to manage allocated resources.
\end{block}
\onslide<2->
\begin{lstlisting}[language={Rust},basicstyle=\ttfamily\Small]
pub struct LibGroonga {/* omitted */}

impl LibGroonga {
    pub fn new() -> Result<LibGroonga, String> {
        // initialize libgroonga
    }

    fn close(&mut self) -> Result<(), String> {
        // finalize libgroonga
    }
}

impl Drop for LibGroonga {
    // Called when a variable goes out of scope.
    fn drop(&mut self) {
        self.close().unwrap();
    }
}
\end{lstlisting}
\end{frame}

\begin{frame}{Cargo's build script mechanism}
\begin{block}{About Cargo\footnote[frame]{http://doc.crates.io/index.html}}
  Cargo, which is the package manager for Rust, has build script feature\footnote[frame]{http://doc.crates.io/build-script.html}.
  \newline
  This feature is used to customize building phase.
  \newline
  Some crates need to link non-Rust code. This kind of linking task sometimes should be customizable.\footnote[frame]{Some crate should compile C libraries before linking. And some crate should distinguish platforms whether it is Windows or not.}
\end{block}
\end{frame}

\begin{frame}{pkg-config}
  \begin{block}{pkg-config}
    pkg-config is the one of great tool on UNIX like environment.
    \onslide<2->
    \newline
    It is a helper tool used when compiling applications and libraries.
  \end{block}
\end{frame}

\begin{frame}[fragile]{pkg-config in Rust}
  \begin{block}{pkg-config crate}
    Rust community already has pkg-config crate! Amazing!!
  \end{block}
  To use pkg-config, specify the following in Cargo.toml:
\begin{lstlisting}[language={XML},basicstyle=\ttfamily\Small]
[package]
  ...
build = "build.rs"

[build-dependencies]
pkg-config = "~0.3"
\end{lstlisting}
\end{frame}

\newcommand\XSmall{\fontsize{7}{7}\selectfont}
\begin{frame}[fragile]{pkg-config in Rust}
  \begin{block}{pkg-config in build script}
    pkg-config is used in build script.
  \end{block}
  \begin{lstlisting}[language={Rust},basicstyle=\ttfamily\XSmall]
/// build.rs
extern crate pkg_config;

use std::env;

fn main() {
    let target = env::var("TARGET").unwrap();

    if !target.contains("windows") {
        if let Ok(info) = pkg_config::probe_library("groonga") {
            if info.include_paths.len() > 0 {
                let paths = env::join_paths(info.include_paths).unwrap();
                println!("cargo:include={}", paths.to_str().unwrap());
            }
            return;
        }
    }
}
  \end{lstlisting}
  Yay, it's so easy!
\end{frame}

\begin{frame}[fragile]{For Windows?}
  \begin{block}{Windows support in build script}
    Why not support Windows?
    \newline
    \onslide<2->
    It's a nightmare for *nix developers.
    \newline
    \onslide<3->
    But Rust community encourages to support Windows.
  \end{block}
\end{frame}

\begin{frame}{Windows support}
  \begin{block}{Windows support in build script}
    If you use Windows, how to set environment information?
    Use pkg-config? No, Windows platform often lacks of it.
    \newline
    \onslide<2->
    Instead, we can always use \textbf {environment variables}.
  \end{block}
\end{frame}

\begin{frame}{Windows support}
  \begin{block}{Actual build script example}
    See the next page.
  \end{block}
\end{frame}

\begin{frame}[fragile]{Windows support in build script}
  \begin{lstlisting}[language={Rust},basicstyle=\ttfamily\XSmall]
// fn main() ..
let target = env::var("TARGET").unwrap();

let lib_dir = env::var("GROONGA_LIB_DIR").ok();
let bin_dir = env::var("GROONGA_BIN_DIR").ok();
let include_dir = env::var("GROONGA_INCLUDE_DIR").ok();

if lib_dir.is_none() && include_dir.is_none() {
    if !target.contains("windows") {
        // same as before
    }
}

let lib = "groonga";
let mode = if env::var_os("GROONGA_STATIC").is_some() {
    "static"
} else {
    "dylib"
};

if let Some(lib_dir) = lib_dir {
    println!("cargo:rustc-link-search=native={}", lib_dir);
}
if let Some(bin_dir) = bin_dir {
    println!("cargo:rustc-link-search=native={}", bin_dir);
}

println!("cargo:rustc-link-lib={}={}", mode, lib);

if let Some(include_dir) = include_dir {
    println!("cargo:include={}", include_dir);
}
  \end{lstlisting}

\end{frame}

\frame{\centering \Large Demo}

\begin{frame}{Conclusion}
  \begin{itemize}
  \item rust bindgen makes easy to create binding.
  \item Drop trait is useful to manage allocated resources.
  \item cargo package manager can handle custom build script.
  \item cargo's build script can handle environment variables which is often used for Windows platform.
  \item Rust bindings sometimes works well on Windows.
  \end{itemize}
\end{frame}

\frame{\centering \Large Any questions?}

\end{document}
